\documentclass[11pt]{article}
\usepackage[a4paper, margin = 1in]{geometry}
\usepackage[doublespacing]{setspace}
\frenchspacing
\usepackage{parskip}
\setlength{\parindent}{15pt}
\setlength{\parskip}{0pt}
\usepackage{amsmath}
\usepackage{amssymb}
\usepackage{textcomp}
\usepackage{csquotes}

\begin{document}
We apply the ordered probit model. We assume people's gun control attitude is a latent variable \(y^* \sim \mathcal{N}(\mu, \sigma^2)\).\footnote{If we assume \(y^*\) follows a logistic distribution, which has heavier tails than a normal distribution, we will choose the ordered logit model. There should not be a practical difference produced by different link functions.} What we observe instead is their revealed attitude when answering the following question (denoted as \(y\)).
\begin{displayquote}
\itshape
Federal law should ban the possession of handguns except by law enforcement personnel. Indicate your level of agreement with the statement.\\
\end{displayquote}
\[
    y = 
    \begin{cases}
    \text{strongly disagree} \quad \text{if} \quad y^* \leq \tau_{1},\\
    \text{disagree} \quad \text{if} \quad \tau_{1} < y^* \leq \tau_{2},\\
    \text{netural} \quad \text{if} \quad \tau_{2} < y^* \leq \tau_{3},\\
    \text{agree} \quad \text{if} \quad \tau_{3} < y^* \leq \tau_{4},\\
    \text{strongly agree} \quad \text{if} \quad \tau_{4} < y^*,
    \end{cases}
\]
in which \(\tau_{1}, \tau_{2}, \tau_{3}, \tau_{4}\) are cutoff points. To estimate the conditional probability of a certain revealed gun control attitude, we have
\begin{align*}
    &\text{Pr}(y = \text{strongly disagree}|\mathbf{x}) = \Phi(\tau_{1} - \mathbf{x^{\prime}}\boldsymbol{\beta}),\\
    &\text{Pr}(y = \text{disagree}|\mathbf{x}) = \Phi(\tau_{2} -\mathbf{x^{\prime}}\boldsymbol{\beta}) - \Phi(\tau_{1} - \mathbf{x^{\prime}}\boldsymbol{\beta}),\\
    &\vdots\\
    &\text{Pr}(y = \text{strongly agree}|\mathbf{x}) = 1 - \Phi(\tau_{4} - \mathbf{x^{\prime}}\boldsymbol{\beta}),
\end{align*}
in which \(\Phi\) is standard normal distribution's cumulative density function and \(\mathbf{x}\) is a row vector of independent variables.

For the panel design, we observe \(y_{i, t}\), in which \(i = 1, 2, \ldots, n\) denoting survey respondents and \(t = before\_event, after\_event\). If \(y_{i, t}\) is continuous (like in the article), then putting the first difference \(\Delta y_{i, t} = y_{i, before\_event} - y_{i, after\_event}\) at the left-hand side will give us a within unit (\(i\)) estimate of the event's effect. However, if we treat \(y_{i, t}\) as a categorical variable, then \(\Delta y_{i, t}\) hardly makes sense. For instance, we do not know what is the meaning of strongly agree minus agree. One way to fix this problem is that we could see the direction of change. Specifically, after coding strongly disagree to strongly agree as 1 to 5, we have
\[
\Delta y_{i, t}^{relative\_change} = 
\begin{cases}
\text{gun-control favourable change} \quad \text{if} \quad \Delta y_{i,t} > 0,\\
\text{unchanged} \quad \text{if} \quad \Delta y_{it} = 0,\\
\text{gun-control opposite change} \quad \text{if} \quad \Delta y_{i,t} < 0,
\end{cases}
\]
and then, we are able to proceed using ordered probit model.

Another problem is by using the first difference, we can only follow the regression-on-each-subgroup approach instead of the interaction term approach as we use for the cross-sectional design.
\end{document}