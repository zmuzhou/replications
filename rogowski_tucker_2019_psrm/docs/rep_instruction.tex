\documentclass[11pt]{article}
\usepackage{/Users/mz/Desktop/modules/style_main}

\begin{document}

\texttt{repcode.R} replicates the numerical results underyling figure 2 in the original article.
The coefficients and standard errors estimates are identical.
It corresponds to line 20 to 143 in the \texttt{.do} file. See the \texttt{.log} file for the corresponding \texttt{Stata} output. 

The authors used comparison-of-means while we use the regression framework. 
For the left panel of figure 2, the comparison-of-means compares \(\mu_{\text{preevent}}\) and \(\mu_{\text{postevent}}\). The regression is \(y = \alpha + \beta \times \text{postevent} + \epsilon\), where \(\text{postevent} = 1\) for respondents who completed the survey after the event. \(\beta\) is the estimate of interest. \texttt{tm\_finish} documents the time when respondents finished the December 2012 survey. Few of them finished in January 2013 but they still answered the December version survey so they are counted as well.

For the right panel of figure 2, the comparison-of-means compares \(\mu_{\text{dec}}\) and \(\mu_{\text{jan}}\). The regression is \(y_{\text{jan}} - y_{\text{dec}} = \alpha + \epsilon\). The intercept, \(\alpha\), is the estimate of interest. Respondents who finished the December survey after the event are excluded in this estimation.

Heteroscedasticity-robust standard errors are employed in all analyses.

\end{document}



